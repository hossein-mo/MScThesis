\chapter{نتیجه‌گیری}
\label{ch5} 

در فصل \ref{ch2} روشی به نام آزمون چپمن-کولموگروف معرفی شد که برای تخمین حافظه یک فرایند تصادفی یا 
همان طول مارکوف استفاده می‌شود و دیدیم که این روش به خوبی حافظه سری‌های زمانی تولید شده با مدل خودبرگشت را تخمین می‌زند. 
در ادامه در فصل \ref{ch3} آزمون چپمن-کولموگروف را برای تخمین حافظه و تخمین وابستگی دو فرایند غیرمارکوف وابسته تعمیم دادیم. 
به منظور آزمایش این روش به مدلی برای تولید سری‌های زمانی وابسته با طول‌های مارکوف دلخواه نیاز بود که برای این کار با تعمیم مدل 
خودبرگشت و اضافه کردن جملاتی برای وابسته کردن دو سری زمانی به یکدیگر موفق شدیم سری‌های زمانی وابسته با طول‌های مارکوف 
دلخواه تولید کنیم. در ادامه نشان دادیم که تعمیم آزمون چپمن-کولموگروف برای فرایندهای غیرمارکوف وابسته به خوبی طول مارکوف سری‌های زمانی 
وابسته تولید شده با استفاده از مدل ارائه شده را تخمین می‌زند همچنین وابستگی این دو سری زمانی به یکدیگر را به خوبی نشان می‌دهد. 

آزمون چپمن-کولموگروف برای دو فرانید غیرمارکوف وابسته به دو شکل بر روی سری‌های زمانی تولید شده با استفاده 
از مدل ارائه شده بررسی شد؛ در حالت اول تعدادی زیادی نمونه مستقل از یکدیگر تولید شد. در این حالت می‌توانیم وابستگی 
سری‌های زمانی به خودشان و یگدیگر را در دو زمان متفاوت بررسی کنیم. اما از آنجایی که اغلب در دنیای واقعی ممکن نیست
 که بیش از یک سری زمانی از فرانیدهای تصادفی داشته باشیم. بنابراین با این فرض که سری‌های زمانی 
  مانا هستند توانستیم آزمون چپمن-کولموگروف را بر حسب فاصله زمانی یعنی $\tau$ انجام دهیم.

 در ادامه در فصل \ref{ch4} با استفاده از آزمون چپمن-کولموگروف تلاش کردیم تا حافظه و وابستگی دو رمزارز بیتکوین و اتریوم را بررسی کنیم. 
 برای این کار از قیمت قراردادهای دائمی بیتکوین و اتریوم در صرافی بیتمکس که بیشترین حجم معاملات را در میان 
 صرافی‌های دیگر دارد استفاده کردیم. با توجه به اینکه رفتار بازارهای مالی می‌تواند وابسته به زمان باشد بنابراین محاسبه طول مارکوف را در 
 پنجره‌های هفتگی انجام دادیم که در این پایان‌نامه نتایج سه تا از این پنجره‌ها آورده شده‌اند. با توجه به \ref{fig:ckxbteth1} یعنی جایی که 
 نوسان قیمت بیتکوین در بیشترین مقدار خود در این بازه سه ماهه بوده می‌بینیم که طول مارکوف بیتکوین و اتریوم طول‌های نسبتا بزرگ و از مرتبه 
 چند دقیقه هستند. اما از طرفی در بازه شماره ۲ یعنی شکل \ref{fig:ckxbteth2} یعنی جایی که نوسان قیمت در کمترین 
 حالت خود بوده است می‌بینیم که طول مارکوف خیلی کوتاه است. 
 این نتایج می‌تواند به این معنی باشد که اتفاقات بزرگ و مهم در این بازار حافظه‌دار و وابسته‌اند و 
 در مواقعی که اتفاق مهمی در بازار رخ نمی‌دهد رفتار این دو رمزارز تقریبا مستقل و بدون حافظه است. اما از طرف دیگر باید توجه 
 داشت که هرچند طول مارکوف با بزرگ شدن نوسان افزایش می‌یابد اما این مساله لزوما به این معنی نیست که هرچه بازار نوسان بزرگ‌تری داشته باشد 
 طول مارکوف بیشتری هم خواهد داشت. نتایج شکل \ref{fig:ckxbteth3} می‌تواند نشان‌دهنده این 
 موضوع باشد. طول مارکوف در این بازه 
 از دو بازه دیگر بیشتر است در حالی که نوسان بیتکوین از بازه شماره ۱ کمتر است. 

 همانطور که دیدیم آزمون چپمن-کولموگروف تعمیم یافته با دقت خوبی طول مارکوف دو فرایند غیرمارکوف وابسته را تخمین می‌زند. از آنجایی که 
 در بازارهای مالی، بسته به نوع بازار ممکن است سهام یا جفت ارزهای زیادی در حال معامله باشند بنابراین شاید بتوانیم با بررسی هر جفت 
 سهام یا ارز با آزمون چپمن-کولموگروف تعمیم یافته دید کلی نسبت به ساختار آن بازار به دست آورد و مشاهده تغییرات این ساختار در 
 طول زمان ممکن است اطلاعات جدیدی در مورد دینامیک بازار در اختیار ما قرار دهد و بینش ما را نسبت به بازار بهبود بدهد.



