\begin{latin}
\newpage
\thispagestyle{empty}
\noindent
\centerline{\textbf{\large{Abstract}}}
\\
There are many systems in the real world showing stochastic behaviors. We use stochastic processes to model such behaviors. Since no system evolves entirely independent and interacts with other systems, their individual study could lead to vague results. In addition to such interdependency, a system may also depend on its own past, or in other words, it has memory. In the context of stochastic processes a process with memory is called a non­-Markov process. Financial markets are an example of complex systems whose behavior can be modeled using stochastic processes with memory; they interact with each other in a sense that they are influenced by each other past and also themselves. Given the role of complex systems such as financial markets in human life, it is necessary to have methods for estimating memory and the dependency of stochastic processes. One of the most common methods for estimating the memory of a non-Markov process (Markov length) is the Chapman-Kolmogorov test. This dissertation aims to generalize the Chapman-Kolmogorov test into two dependent non-Markov processes. To show the efficiency in estimating the Markov length for two dependent non-Markov processes, we need a model that generates two dependent time series with the desired Markov lengths. The results show that the generalized Chapman-Kolmogorov test accurately estimates Markov lengths in two dependent non-Markov processes. In the end, we analyzed Bitcoin and Ethereum, which are cryptocurrencies. Results show that significant moves have memory, but small ones are independent of each other.

\textbf{Keywords:}

\textit{non-Markov dependent processes, Chapman-Kolmogorov test, Memory, Markov length, Autoregressive model, Cryptocurrency, Bitcoin, Ethereum}
\end{latin}