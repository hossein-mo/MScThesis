\thispagestyle{empty}
\noindent
\centerline{\textbf{\large{چکیده}}} \\
در دنیای واقعی سیستم‌هایی وجود دارند که دارای رفتار تصادفی هستند و برای مدل‌سازی آن‌ها از فرایندهای تصادفی استفاده می‌کنیم.
از آنجایی که هیچ سیستمی کاملا مستقل نیست و با دیگر سیستم‌ها برهمکنش دارد در تحلیل و بررسی این سیستم‌ها بهتر است وابستگی 
آن‌ها به یکدیگر را در نظر داشت. در کنار این وابستگی یک سیستم می‌تواند به گذشته خودش نیز وابسته باشد 
یا به عبارت دیگر حافظه‌دار باشد. در فرایندهای تصادفی اگر فرایندی حافظه‌دار باشد به ‌آن فرایند غیرمارکوف گفته می‌شود.
به عنوان مثال بازارهای مالی به عنوان نمونه‌ای از سیستم‌های پیچیده هستند که رفتار آن‌ها را می‌توان با فرایندهای تصادفی 
که به یکدیگر وابسته و غالبا به گذشته خود نیز وابسته هستند مدل‌سازی کرد.
با توجه به نقش سیستم‌های پیچیده از قبیل بازارهای مالی در زندگی بشر وجود روش‌هایی برای تخمین حافظه 
و وابستگی فرایندهای تصادفی به یکدیگر ضروری است.
یکی از روش‌های مرسوم برای تخمین حافظه یک فرایند غیرمارکوف(طول مارکوف)، آزمون چپمن-کولموگروف است.
هدف از این پایان‌نامه تعمیم آزمون چپمن-کولموگروف به دو فرایند غیرمارکوف وابسته است.
در ادامه به منظور نشان دادن کارایی در تخمین طول مارکوف برای دو فرایند غیرمارکوف وابسته نیازمند مدلی هستیم که 
سری‌های زمانی وابسته با طول‌های مارکوف دلخواه را تولید کند. بدین منظور مدل خودبرگشت را تعمیم می‌دهیم. 
نتایج نشان می‌دهد که آزمون چپمن-کولموگروف تعمیم یافته با دقت بالایی طول‌های مارکوف را در دو فرایند 
غیرمارکوف وابسته تخمین می‌زند.
در انتها با این روش به بررسی دو رمزارز بیتکوین و اتریوم پرداخته می‌شود و نتایج نشان می‌دهد که نوسانات بزرگ از لحاظ حافظه 
به یکدیگر وابسته هستند ولی نوسانات کوچک از یکدیگر مستقل هستند.

\textbf{کلمات کلیدی:}

\textit{فرایند‌های غیر مارکوف وابسته، آزمون چپمن-کولموگروف، حافطه، طول مارکوف، مدل خودبرگشت، رمزارز، بیتکوین، اتریوم}