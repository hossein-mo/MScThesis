% \newpage
\chapter*{پیشگفتار}
\markboth{پیشگفتار}{پیشگفتار}
\label{intro} 
در طول تاریخ انسان‌ها همیشه در تلاش بوده‌اند تا محیط پیرامون خود را بهتر بشناسند و 
رفته رفته این تلاش‌ها باعث پیدایش علوم مختلف شده است. در این میان به دلیل محدود بودن توانایی انسان‌ها و همچنین برخی 
محدودیت‌های بنیادی باعث شده تا انسان‌ها دست به گسترش مباحثی بزنند که می‌تواند به شکلی بر این محدودیت‌ها غلبه کنند. 
به عنوان مثال مکانیک آماری یکی از مواردی است که برای بررسی سیستم‌های بس ذره‌ای استفاده می‌شود؛ سیستم‌هایی که 
حل دقیق آن‌ها برای ما ممکن نیست. یکی دیگر از مباحثی که برای غلبه بر این ناتوانی استفاده می‌شود مبحث 
فرایندهای تصادفی است.برخی سیستم‌ها به دلیل رفتار شبه تصادفی که دارند نمی‌توان آن‌ها را به شکل دقیق بررسی کرد. 
اما با استفاده از فرایندهای تصادفی می‌توان رفتار این سیستم‌ها را مدل کرد.
فیزیک\cite{paul_stochastic_2013}، شیمی\cite{kampen_stochastic_2007}، بیولوژی\cite{bressloff_stochastic_2014} و پردازش سیگنال\cite{dougherty_random_1999} 
 از جمله مواردی هستند که فرایندهای تصادفی به شکل گسترده در آن‌ها استفاده می‌شود.
بازارهای مالی یکی دیگر از مواردی است که برای تحلیل و بررسی آن از فرایندهای تصادفی استفاده می‌شود. \cite{steele_stochastic_2001,shreve_continuous-time_2008}

در دنیای واقعی بررسی سیستم‌ها به شکل مستقل و بدون در نظر گرفتن 
برهم‌کنش با دیگر سیستم‌ها اطلاعات جامع و کاملی از تحول این سیستم‌ها 
به ما نمی‌دهد ممکن است سیستمی با سیستم‌های دیگر برهم‌کنشی 
ضعیف یا حتی قابل صرف نظر کردن داشته باشد اما 
در حقیقت هیچ سیستمی در واقعیت کاملاً مستقل نیست. 
در بسیاری از موارد بررسی سیستم‌ها به شکل مستقل 
می‌تواند از درک درست ساز و کار سیستم مورد مطالعه جلوگیری کند و 
در صورتی که هدف پیشبینی آینده یک سیستم دینامیکی باشد بررسی آن به شکل 
مستقل می‌تواند باعث بروز خطا در پیشبینی بشود.
از بحران مسکن آمریکا در سال 2008 بارها به 
عنوان یک مثال مهم از اهمیت اثرگذاری بازارها بر یکدیگر نامبرده شده است، این بحران که از 
سال 2005 شروع و تا سال 2008 ادامه پیدا کرد، اثری مخرب بر دیگر بازارهای مالی آمریکا 
و دیگر نقاط دنیا داشت و باعث به وجود آمدن 
یک رکود اقتصادی در دنیا شد، که نشان دهنده 
اهمیت ارتباط و اثر گذاری بازارهای مالی بر یکدیگر است.\cite{islam_great_2011, 7344854}
بنابراین باید تلاش کرد تا سیستم‌ها را به شکل وابسته با 
یکدیگر در نظر بگیریم تا بتوان تحول این سیستم‌ها را به شکلی دقیق بررسی کنیم.

نکته مهم دیگر در مورد سیستم‌ها و فرایندهای واقعی وابستگی آن‌ها به حالت‌های گذشته 
خودشان است که اصطلاحا به آن‌ها سیستم‌ها یا فرایندهای حافظه‌دار یا غیرمارکوف گفته می‌شود. در اینگونه سیستم‌ها حالت‌های گذشته یک سیستم در آینده آن 
تاثیر گذاز است. تاکنون مدل‌های مختلفی برای سیستم‌های حافظه‌دار ارائه شده است. مثلا مدل خودبرگشت\LTRfootnote{Autoregressive model} 
یکی از ساده‌ترین مدل‌هایی است که می‌توان با آن سری‌های زمانی حافظه‌دار را تولید و بررسی کرد.\cite{shumway2011time, GANDHI20157}
مدل‌‌های بسیار زیادی از این دست وجود دارند که با تعیین پارامترهای آن‌ها می‌توان سری‌های زمانی حافظه‌دار تولید کرد یا با تخمین پارامترها 
از روی سری‌های زمانی واقعی مانند بازارهای مالی می‌توان سیستم‌ها و فرایندهای واقعی را بررسی کرد.\cite{shumway2011time} 
از جمله مدل‌های دیگری که از‌ آن‌ها در بررسی سری‌های زمانی حافظه‌دار استفاده می‌شود می‌توان به مدل 
میانگین متحرک خودبرگشت یکپارچه اشاره کرد که به طور گسترده‌ای نیز در تحلیل و پیشبینی بازارهای مالی استفاده شده است\LTRfootnote{Autoregressive integrated moving average}.
\cite{mills_time_1990, asteriou_applied_2011, box_time_2016}.

روش‌های مختلفی برای تخمین حافظه مانند محاسبه خودهمبستگی\LTRfootnote{Autoorrelation} وجود دارد.\cite{stroe-kunold_estimating_2009} 
در این میان روش‌های دیگری نیز وجود دارد که می‌توانند تخمینی از حافظه یا طول مارکوف فرایندها به دست 
آورند، از جمله
روش‌هایی که با آن طول مارکوف 
یک فرایند را تخمین می‌زنند می‌توان به روش  $\chi^2$\cite{ghasemi_markov_2007}، 
روش ویلکاکسون\LTRfootnote{Wilcoxon test}\cite{waechter_stochastic_2004} و 
آزمون چپمن-کولموگروف\LTRfootnote{Chapman-Kolmogorov test}\cite{fazeli_probing_2008,risken_fokker-planck_1984,friedrich_approaching_2011,friedrich_description_1997} اشاره کرد.
در این میان آزمون چپمن-کولموگروف به دلیل اینکه با استفاده از توزیع احتمال محاسبه می‌شود می‌توان با تعمیم آن وابستگی یک فرایند 
به فرایند دیگر را نیز آزمود تا بتوان برای دو فرایند وابسته طول مارکوف و وابستگی را  تخمین زد. در این پایان‌نامه 
آزمون چپمن-کولموگروف را برای دو فرایند غیرمارکوف وابسته تعمیم می‌دهیم تا غیر از تخمین طول حافظه بتوانیم طول وابستگی 
دو فرایند را تخمین بزنیم.

ابتدا در فصل اول به توضیح مقدمات احتمال و مفاهیم مورد نیاز احتمالات برای تعمیم آزمون چپمن-کولموگروف خواهیم پرداخت. 
در این فصل مباحث مقدماتی احتمال از جمله تعریف توزیع‌های احتمال توأم و شرطی توضیح داده شده‌اند. 
در فصل دوم تلاش کردیم تا مفاهیم کلی و مقدمات مورد نیاز از فرایندهای تصادفی را بیان کنیم. در انتهای این فصل 
فرایندهای غیرمارکوف را تعریف می‌کنیم و روش‌های محاسبه طول حافظه که به آن طول مارکوف می‌گوییم را 
معرفی می‌کنیم. در فصل سوم به کار خودمان می‌پردازیم و آزمون چپمن-کولموگروف را 
برای دو فرایند غیرمارکوف وابسته تعمیم خواهیم داد. برای اینکه ببینیم آزمون چپمن-کولموگروف تعمیم داده شده 
می‌تواند به درستی نتیجه بدهد، مدل خودبرگشت را برای تولید دو فرایند غیرمارکوف وابسته تعمیم می‌دهیم و آزمون تعمیم 
یافته را روی داده‌های این مدل آزمایش می‌کنیم. در فصل ۴ نیز با استفاده از آزمون تعمیم یافته طول مارکوف دو رمز ارز 
بیتکوین و اتریوم را برای چند بازه زمانی محاسبه می‌کنیم. در نهایت در فصل آخر نیز به جمع‌بندی و نتیجه‌گیری نتایج به دست آمده خواهیم پرداخت.




% \section{جفت‌شدگی در بازارهای مالی}